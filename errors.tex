\documentclass[letterpaper, 10pt]{article}
\usepackage{amsmath, amsthm, amssymb}
\usepackage{mathtools}
\usepackage{enumitem}
\usepackage{listings}

\lstset{language=C, keywordstyle={\bfseries}}
\newcommand{\lin}[1]{\lstinline{#1}}
\setlength\parindent{0pt}
\setlength\parskip{5pt}

\begin{document}
\section{Error handling}
Variable named \textbf{errno} stores error codes, which are defined in a header file \lstinline{<error.h>}. \lstinline{perror()} prints a string you pass to it,
followed by a colon with space and then the textual representation of the current value of \lstinline{errno}. \lin{strerror()} return a string representation
of current \lin{errno} value.

Don't forget to output error messages to \lin{stderr} file stream.

Exiting a program and setting exit status: \lin{EXIT_SUCCESS} on success and \lin{EXIT_FAILURE} on failure.

\end{document}
